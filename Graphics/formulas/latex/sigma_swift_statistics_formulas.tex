\documentclass[a4paper,twoside, 11pt]{article}

% Packages
% ---------------------

\usepackage{amsmath} % Needed for command eqref
\usepackage{amssymb} % For math symbols like R \mathbb{R}
\usepackage{siunitx} % Physical numbers and units
\usepackage{mathtools}

% Page settings
% ---------------------

\usepackage[top=2cm, bottom=1.8cm,left=2.5cm,right=2.5cm]{geometry} %  Page margins
\setlength{\parskip}{\baselineskip} % Add space between paragraphs
\parindent=0cm % Remove the paragraph indent of the first line
\addtolength{\jot}{2\jot} % Double the line between equations

% Set vertical space around equations.
\AtBeginDocument{%
 \abovedisplayskip=10pt plus 3pt minus 9pt
 \abovedisplayshortskip=6pt plus 3pt
 \belowdisplayskip=10pt plus 3pt minus 9pt
 \belowdisplayshortskip=5pt plus 3pt minus 4pt
}


\begin{document}

\subsection*{Skewness A}

\begin{flalign*}
&\frac{n}{(n-1)(n-2)}\sum_{i=1}^{n} \frac{(x_i - \bar{x})^3}{s^3},&
\end{flalign*}
where $n$ is the sample size, $\bar{x}$ is the sample mean and\\$s$ is the sample standard deviation.

% ------------------
%
% Formulas for "Programming harmonic oscillator" blog post
% http://evgenii.com/blog/programming-harmonic-oscillator
%
% Images are made from Skim PDF viewer at 7th magnification.
%
% ------------------


% \[
%     \vec{F} = -k \vec{x}
% \]

% \begin{align*}
%     ma &= -kx\\
%     a &= -\frac{k}{m}x\\
%     \ddot{x} &= -\frac{k}{m}x\\
%     v_{\textrm{new}} &= v_{\textrm{current}} + \textrm{deltaT} \times \textrm{acceleration}
% \end{align*}

% \[
%     x(t) = A\cos(\omega t) + B \sin(\omega t)
% \]

% \[
%     \omega = \sqrt{\frac{k}{m}}
% \]

% \begin{align*}
%     \log_{a^k}x &= \frac{1}{k}\log_a x\\
%     \log_{a}x &= \frac{1}{\log_x a}
% \end{align*}


% \[
%   f(t) = -\big[e^{-\frac{t}{\textrm{a}}} \cdot \cos(t \cdot w)\big] + 1
% \]

% \section*{Calculating surface temperatures of Venus, Earth and Mars}


% \begin{align*}
%     T &= \Big( \frac{L}{16 \pi d^2 \sigma} \Big)^{\frac{1}{4}} \tag{Planet's surface temperature equation}\\
%     \textrm{Where}\\
%     L &= \SI{4E26}{\joule \second^{-1}} \tag{Sun's luminocity}\\
%     \sigma &= \SI{5.67E-8}{\joule \meter^{-2} \second^{-1} \kelvin^{-4}} \tag{Stephan-Boltzmann constant}\\
%     d&: \textrm{Distance from planet to the Sun}
% \end{align*}


% \subsection*{Surface temperature of Venus}

% \begin{align*}
%     d_{\textrm{venus}} &= \SI{1.08E11}{\meter^{-1}}\\
%     T &= \Big( \frac{L}{16 \pi d^2 \sigma} \Big)^{\frac{1}{4}}\\
%      &= \Big( \frac{\SI{4E26}{\joule \second^{-1}}}{16 \pi \cdot (\SI{1.08E11}{\meter^{-1}})^2 \cdot \SI{5.67E-8}{\joule \meter^{-2} \second^{-1} \kelvin^{-4}}} \Big)^{\frac{1}{4}}\\
%      &= \SI{331}{\kelvin} \approx \SI{58}{\celsius}.
% \end{align*}
% The surface temperature of Venus is $\SI{58}{\celsius}$.

% \subsection*{Surface temperature of Earth}

% \begin{align*}
%     d_{\textrm{earth}} &= \SI{1.5E11}{\meter^{-1}}\\
%     T &= \Big( \frac{L}{16 \pi d^2 \sigma} \Big)^{\frac{1}{4}}\\
%      &= \Big( \frac{\SI{4E26}{\joule \second^{-1}}}{16 \pi \cdot (\SI{1.5E11}{\meter^{-1}})^2 \cdot \SI{5.67E-8}{\joule \meter^{-2} \second^{-1} \kelvin^{-4}}} \Big)^{\frac{1}{4}}\\
%      &= \SI{281}{\kelvin} \approx \SI{8}{\celsius}.
% \end{align*}
% The surface temperature of Earth is $\SI{8}{\celsius}$.

% \subsection*{Surface temperature of Mars}

% \begin{align*}
%     d_{\textrm{mars}} &= \SI{2.28E11}{\meter^{-1}}\\
%     T &= \Big( \frac{L}{16 \pi d^2 \sigma} \Big)^{\frac{1}{4}}\\
%      &= \Big( \frac{\SI{4E26}{\joule \second^{-1}}}{16 \pi \cdot (\SI{2.28E11}{\meter^{-1}})^2 \cdot \SI{5.67E-8}{\joule \meter^{-2} \second^{-1} \kelvin^{-4}}} \Big)^{\frac{1}{4}}\\
%      &= \SI{228}{\kelvin} \approx \SI{-45}{\celsius}.
% \end{align*}
% The surface temperature of Mars is $\SI{-45}{\celsius}$.

% \[
%     \frac{1}{\sqrt{2 \sigma^2 \pi}} e^{ - \frac{(x - \mu)^2}{2 \sigma^2} }
% \]

% \[
%     4x^3 \sec{x^4}
% \]


\end{document}